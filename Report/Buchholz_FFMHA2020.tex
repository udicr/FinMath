\documentclass[11pt,oneside,a4paper]{article}
\usepackage{graphicx}
\usepackage{booktabs}
\usepackage{caption}
\usepackage{subcaption}
\usepackage{amsmath}
\usepackage{amsfonts}
\usepackage{amssymb}
\usepackage{lscape}
\usepackage{psfrag}
\usepackage{mathtools}
\usepackage[usenames]{color}
\usepackage{bbm}
\usepackage[update]{epstopdf}
\usepackage[bookmarks,pdfstartview=FitH,a4paper,pdfborder={0 0 0}]{hyperref}
\usepackage{verbatim}
\usepackage{listings}
\usepackage{textcomp}
\usepackage{fancyhdr}
\usepackage{multirow}
\pagestyle{fancy}
\usepackage{tikz}
\usepackage{course}
\usepackage{csvsimple}
\usepackage{float}

\renewcommand{\sectionmark}[1]{\markboth{#1}{#1}}
\renewcommand{\subsectionmark}[1]{\markright{#1}}

\fancyhf{}
\fancyhead[RO]{\nouppercase{\footnotesize\sc\leftmark\ \hrulefill\ \thepage}}
\fancyhead[RE]{\nouppercase{\footnotesize\sc\thepage\ \hrulefill\ }}
\renewcommand{\headrulewidth}{0pt}

\makeatletter
\def\cleardoublepage{\clearpage\if@twoside \ifodd\c@page\else%
	\hbox{}%
	\thispagestyle{empty}%
	\clearpage%
	\if@twocolumn\hbox{}\clearpage\fi\fi\fi}
\makeatother


\renewcommand{\topfraction}{0.9}  % max fraction of floats at top
\renewcommand{\bottomfraction}{0.8} % max fraction of floats at bottom
% Parameters for TEXT pages (not float pages):
\setcounter{topnumber}{2}
\setcounter{bottomnumber}{2}
\setcounter{totalnumber}{4}            % 2 may work better
\setcounter{dbltopnumber}{2}           % for 2-column pages
\renewcommand{\dbltopfraction}{0.9}    % fit big float above 2-col. text
\renewcommand{\textfraction}{0.07}     % allow minimal text w. figs
% Parameters for FLOAT pages (not text pages):
\renewcommand{\floatpagefraction}{0.7}  % require fuller float pages
% N.B.: floatpagefraction MUST be less than topfraction !!
\renewcommand{\dblfloatpagefraction}{0.7} % require fuller float pages

\sloppy

\widowpenalty=10000
\clubpenalty=10000

\edef\today{%\number\day\
	\ifcase\month\or
	January\or February\or March\or April\or May\or June\or July\or
	August\or September\or October\or November\or December\fi\ \number\year}
\title{\vspace*{40.0mm}
	\bf\sf Design of a structured product
	\vspace*{20.0mm} \\
	\vspace*{40.0mm}
	%\vspace{-20mm}\framebox{DRAFT VERSION}\vspace{20mm} \\
	% \Large\bf\sf Design of a structured product \vspace*{20.0mm}}
}
\author{\sf Tim Buchholz}
\date{\sf 21.10.20}

\begin{document}
	
	\begin{figure}
		\parbox[t]{125mm}{
			\vspace*{6mm}
			\scriptsize\sf           FACULTY OF ENGINEERING SCIENCE\\
			\scriptsize\sf\bfseries  Fundamentals of Financial Mathematics \\
			\scriptsize\sf           Home Assignment 2020-2021 \\
			\scriptsize\sf           Tim Buchholz}
		\parbox[t]{40mm}{
			\begin{flushright}
				\includegraphics[height=15mm]{logo-eps-converted-to.pdf}
		\end{flushright}}
	\end{figure}
	
	\maketitle
	\thispagestyle{empty}
	\raggedbottom
	
	\cleardoublepage
	\pagenumbering{roman}
	\setcounter{tocdepth}{2}
	
	\tableofcontents
	
	\cleardoublepage
	\pagenumbering{arabic}
	
	\section{Introduction, Assumptions and relevant market data}
	In this assignment the design of two structured products for an underlying stock will be discussed. Namely the underlying stock is Advanced Micro Devices, Inc. , which is also known as AMD. AMD is an American multinational semiconductor company, which produces Computer processors and related technologies for business and consumer markets. 
	For many investors AMD is a really interesting company, as it is beside of Intel (Intel Corp. v. Advanced Micro Devices) the biggest producer for computer processors with even growing partial of the market. 
	Additionally the recent release of the new graphics processing units (GPUs) is very promising as these represent serious competition to Nvidia graphics cards, which dominated the market in the last years.
	
	Following structured products will be introduced:
	\begin{itemize}
		\item Product 1: Partially principal protected note
		\item Product 2: Airbag note
	\end{itemize}
	As underlying Finance instruments we will basically consider three different assets:
	\begin{itemize}
		\item the risk-free bank account with an fixed interest rate $r$
		\item the stock $ S $ (in this case AMD)
		\item Call and Put options on the stock S
	\end{itemize}
	As starting date $ t=0 $ Friday the 20st November is choosen.
	The maturity date of the products is Friday 21st January 2022.
	More precisely the structured product, which will be discussed later, are build on the data presented below, accessed at Friday 20st November at 4:45 pm (MEZ), which is one hour and a quarter after the market opening at that day. The stock price at this moment was $ S_0=85.35 \text{ USD} $. The data is shown in the tables at the end of this section. A small extract of the data is moreover shown in a associated screenshot in the appendix. As the tables already contain the full data, the screenshot only shows the stock price and the first few call options.  \\


	To get a fixed interest rate for the whole time period, the daily treasure yield rate of the US treasure bond at the starting date $ t=0 $ is picked, which we denote by $ r = 0.11 \% $.
	In the appendix also an associated table is shown, which is published at \cite{site_treasure}
    an official website of the United States Government.
   
	
	
	The reason of choice is that the base interest rate of the United States, the United States Fed Funds Rate, is given by a target range of $ 0-0.25 \% $. Moreover the Daily Treasury Yield Curve Rates are given for different time periods exactly. Since the time period of our products is around one year and two months the rate for an investment over one year is picked. As the investment is even longer than one year, that is a rather cautious estimate for the interest rate.
	
	
		\begin{table*}[!ht]
		\csvreader[%
		tabular={|c|c|c|c|c|c|c|},
		table head = \hline\textbf{{LastTradeDate}} & \textbf{Strike} & \textbf{LastPrice} & \textbf{Bid} & \textbf{Ask}  & \textbf{Volume} & \textbf{ImpliedVolatility}\\\hline,
		late after line= \\,
		late after last line=\\\hline %
		]{../data/yahoo_pc_stat_210122_table0_20112020_16:45:35}{LastTradeDate=\LastTradeDate,Strike=\Strike,LastPrice=\LastPrice,Bid=\Bid,Ask=\Ask,Volume=\Volume,ImpliedVolatility=\ImpliedVolatility}%
		{\LastTradeDate &\Strike & \LastPrice & \Bid & \Ask & \Volume & \ImpliedVolatility}
		\centering
		\caption{\label{table1}European Call Options AMD (data extracted from \cite{site_yahoofinance} at 20/11/2020 16:45:35)}
	\end{table*}
	\begin{table*}[!ht]
		\csvreader[%
		tabular={|c|c|c|c|c|c|c|c|},
		table head = \hline\textbf{{LastTradeDate}} & \textbf{Strike} & \textbf{LastPrice} & \textbf{Bid} & \textbf{Ask} & \textbf{Volume} & \textbf{ImpliedVolatility}\\\hline,
		late after line= \\,
		late after last line=\\\hline %
		]{../data/yahoo_pc_stat_210122_table1_20112020_16:45:35}{LastTradeDate=\LastTradeDate,Strike=\Strike,LastPrice=\LastPrice,Bid=\Bid,Ask=\Ask,Volume=\Volume,ImpliedVolatility=\ImpliedVolatility}%
		{\LastTradeDate &\Strike & \LastPrice & \Bid & \Ask & \Volume & \ImpliedVolatility}
		\centering
		\caption{\label{table2}European Put Options AMD (data extracted from \cite{site_yahoofinance} at 20/11/2020 16:45:35)}
	\end{table*}
	\newpage
	\section{Product 1: Partially principal protected note (PPPN)}
	\subsection{Descriptive part: What is a PPPN and why is it interesting?}
	A partially principal protected note (PPPN) is an investment option over an underlying stock, in our case AMD. Many people are initially skeptical when it comes to investing in stocks. The reason for this is that the opportunities come with risks. A PPPN is en investment option, where the risk is partially absorbed and additionally there can even be profit when the stock falls under a specific level. For instance imagine investing $850 \text{ USD}$ in a PPPN with fixed maturity date $ T $ over an underlying stock with initial stock price $ S_0 = 85 \text{ USD} $.\\
	 If one would buy the stock directly this would mean buying ten stocks. If at maturity the stock is up to a value of $ 110 \text{ USD} $ one would have a profit of $ 1100 \text{ USD} - 850 \text{ USD} = 250 \text{ USD}  $. But in the case of $ S_T = 60 \text{ USD} $ one would loose the same amount of money.
	 The PPPN is preventing exactly that: One cannot loose more then $ 10 \% $ of the initial investment. But additionally there is profit if the stock rises above a certain strike $\underline{\text{and}}$ if the stock falls below another fixed strike. The following plot explains what exactly is meant for the example situation above ($ S_0 = 85 \text{ USD}$ and an initial investment of $ 850 \text{ USD} $ with strikes $ K_1 = 60\text{ USD} $ and  $ K_2 = 110\text{ USD} $): 
	\begin{figure}[H]
		\centering
		\includegraphics[width=0.6\linewidth]{payoff_PPPN.jpg}
		 \caption{payoff for an PPPN}
	\end{figure} 
	The yellow line represents the direct investment in the stock, where the blue line represents the total payoff of the PPPN. One can see that there is less risk but one has to pay with a smaller profit in the case of a rising stock and a possible loss of up to $ 10 \% $ if the stock stays close to $ S0 $.  Therefore one can get profit even in the case that the stock goes down.
	\subsection{Concrete product}
	The concrete product which can be offered has the following specifications:
	\begin{itemize}
		\item Since the underlying stock is AMD, consider $ S_0 = 85.35 \text{ USD} $.
		\item The investment can be any multiple of $ N = 85.35 \text{ USD} $ which is the stock price $ S_0 $.
		\item The maturity date is $ 21st $ January $ 2022 $ so the duration $ T = 14 $ months.
		\item As lower strike $ K_1 = 40 \text{ USD} $ is picked.
		\item As higher strike $ K_2 = 130 \text{ USD} $ is picked.
	\end{itemize}
	Then the total payoff can be visualized in the same way as in the descriptive part:
	\begin{figure}[H]
		\centering
		\includegraphics[width=0.8\linewidth]{PPPN_concrete.jpg}
		\caption{payoff for the concrete PPPN offer}
	\end{figure} 
	If the stock goes up $ 100 \% $ till the maturity day one would end up with $ 117.52 \text{ USD}$ for an investment of  $ N = 85.35 \text{ USD} $ . And no matter how the stock moves in the next $ 14 $ months one can never end up with less than $ 76.5 \text{ USD} $.
	\newpage
	\subsection{Technical Part: How does a PPPN work?}
	A PPPN is the combination of three different assets:
	\begin{itemize}
		\item A risk free asset with interest rate $ r $.
		\item Call options on the underlying stock.
		\item Put options on the underlying stock.
	\end{itemize}
	First we have to determine the part of the investment which has to be invested in a risk free way. Therefore we have to discount $ 90 \% $ of the initial investment by the interest rate $ r $ over the time period $ T $. In our case we consider an investment in the US Treasure Bond of 
	\[
		N_{\text{fixed}} = 0.9 N\cdot\exp(-rT) = 0.9  \exp(-0.0011\frac{14}{12}) N = 0.89885 N \ .
	\]
	It remains to invest \[ N_{\text{options}} \leq N-N_{\text{fixed}} = 0.10115N \ . \]
	We are now buying $ \frac{N}{S_0} $ European Put options for the considered maturity date with strike $ K_1 = 40 \text{ USD} $ and the same amount of European Call options for the considered maturity date with strike $ K_2 = 130 \text{ USD} $. To look up the prices of these we have to look up the tables of relevant market data stated above. We can find for the put option \\
	
	\begin{tabular}{|c|c|c|c|c|c|c|}
		\hline\textbf{{LastTradeDate}} & \textbf{Strike} & \textbf{LastPrice} & \textbf{Bid} & \textbf{Ask} & \textbf{Volume} & \textbf{ImpliedVolatility}\\\hline
		2020-11-19 3:08PM EST & 40.0 & 1.42 & 1.35 & 1.5 & 152 & 53.05\%
		\\\hline
	\end{tabular}\\
 $\newline$
	and for the Call option
   
	\begin{tabular}{|c|c|c|c|c|c|c|}
		\hline\textbf{{LastTradeDate}} & \textbf{Strike} & \textbf{LastPrice} & \textbf{Bid} & \textbf{Ask} & \textbf{Volume} & \textbf{ImpliedVolatility}\\\hline
		2020-11-19 3:57PM EST & 130.0 & 6.95 & 6.6 & 6.95 & 457 & 49.87\%
		\\\hline
	\end{tabular}
  
	Since we chose $ N $ as a multiple of the initial stock price $ S_0 $ the ratio $ a = \frac{N}{S_0} $ gives us a natural number which is the amount of each call and put options we have to buy.
	With the ask prices of the options we get 
	\begin{align*}
	c_{\text{put,ask}} &= 1.5 \text{ USD} \approx 0.017575 S_0 \\
	c_{\text{cal,ask}} &= 6.95 \text{ USD} \approx 0.081429 S_0  \\
	N_{\text{options}} &= \frac{N}{S_0} 0.017575 S_0 + \frac{N}{S_0} 0.081429 S_0 \approx 0.099004 N
	\end{align*}
	The remaining amount ($ \approx 0.002146 N $) is then a small profit for the bank for selling the option. Let us now have a look at the payoff of those three assets at maturity:
	\begin{itemize}
		\item The risk free asset will end up at $ 0.9\cdot N $.
		\item Each of the $ a $ Put options generates a payoff at maturity of $ \text{payoff}_{\text{put}} = (K_1-S_T)^{+} $.
		\item Each of the $ a $ Call options generates a payoff at maturity of $ \text{payoff}_{\text{call}} = (S_T- K_2)^{+} $.
	\end{itemize}
	Thus the total payoff is given by
	\begin{align*}
		\text{payoff}_{\text{total}} &= 0.9\cdot	N + a \cdot \text{payoff}_{\text{put}} + a \cdot \text{payoff}_{\text{call}} \\ &= 0.9N + a\cdot((K_1-S_T)^+ + (S_T- K_2)^+) \\
		&=
		\begin{cases}
				0.9N + \frac{N}{S_0}(K_1 - S_T) \; &\text{if } S_T < K_1 \\
				0.9N  \; &\text{if } K_1 \leq S_T \leq K_2 \\
				0.9N + \frac{N}{S_0}(S_T-K_2) \; &\text{if }S_T > K_2 \\
		\end{cases} \\
	\end{align*}
	and the bank just delivers to the investor the payoff in every case. 
	For an investment of \[N = 10\cdot S_0 = 853.5 \text{ USD} \] this would give the bank a profit of around $ 1.84 \text{ USD} $ by having nearly no risk following the common assumptions. Additional margin can be achieved if one is able to invest the risk free asset with a slightly higher interest rate or if its possible to get the desired options a little bit under the ask value.
	\newpage
	\section{Product 2: Airbag note (AN)}
	\subsection{Descriptive part: What is a AN and why is it interesting?}
	Another structured product which tries to handle the risk while investing into the stock market is the so-called Airbag note (AN). The main difference to the PPPN is that the risk handling is done a little bit different. Imagine for example now the stock AMD. 	For a PPPN we saw that the investor can only make profit from a falling stock in quite extreme scenarios.
	Since the company did quite well in the last time one would not expect the stock to fall below lets say $ 80 \% $ of its value at time $ t=0 $. How can now an investor get a downside protection when investing in stocks up to a certain level, for instance those $ 80\% $ of the current stock price. The answer is an Airbag note. If the stock falls but ends up above the so-called airbag level the investor is protected and just gets his initial investment back.
	If the stocks price is below the that level, where the investor is almost sure that this won't happen, then no complete downside protection is provided but the investor still gets a little bit more then when he would have invested in the stock directly. More precisely the total payoff at maturity is then 
	\[
		\text{payoff} = \frac{N}{l S_0}\cdot S_T
	\]
	 where $ l S_0 $ is the airbag level $ N $ is again the initial investment, $ S_0 $ the initial stock price and $ S_T $ the stock price at maturity. Since $ 0<l<1 $ (in our case $ l=0.8 $) this is more than the payoff at maturity which a direct investment in the stock would give.
	 In the case of the rising stock the investor is happy, because he gets a partial participation in the stock performance. This means that in this case the payoff at maturity is given by
	 \[
	 	\text{payoff} = N + p \cdot \frac{N}{S_0}(S_T - S_0)
	 \] 
	 with a fixed participation rate $ p $. Of course the participation rate $ p $ will be lower than one, which means in that case the investor will get a little bit less as if he had invested directly in a stock - that is the trade off in reducing the risk. But in contrast to the PPPN there is no scenario where the stock goes up and the investor will not also make profit from the airbag note.
	 As before we can plot the payoff in dependence of $ S_T $:
	 \begin{figure}[H]
	 	\centering
	 	\includegraphics[width=0.6\linewidth]{payoff_AN.jpg}
	 	\caption{payoff an example AN with $ p = 0.6 $}
	 \end{figure}
	Again the yellow line represents the direct investment in the underlying stock and the dashed verticle lines are representing the initial stock price $ S_0 $ and the airbag level $ A = 0.8 \cdot S_0 $.
	\subsection{Concrete Product}
	The concrete product which can be offered has the following specifications:
	\begin{itemize}
		\item The investment can be any multiple of $ N = 1707 \text{ USD} $. If the bank is capable of buying and selling partial options, for instance if it is expected that the product is sold often and the bank can handle multiple customers together, also less artificial numbers for $ N $ are possible. 
		\item Since the underlying stock is AMD, consider $ S_0 = 85.35 \text{ USD} $.
		\item The investment can be any multiple of $ N = 85.35 \text{ USD} $ which is the stock price $ S_0 $.
		\item The maturity date is $ 21st $ January $ 2022 $ so the duration $ T = 14 $ months.
		\item As Airbag level $ A = l \cdot S_0 = 0.8 \cdot S_0  = 68.28 \text{ USD} $ is picked.
		\item The participation rate which can be offered is $ p = 55 \% $ .
	\end{itemize}
	Then the total payoff can be visualized in the same way as in the descriptive part:
	\begin{figure}[H]
		\centering
		\includegraphics[width=0.8\linewidth]{AN_concrete.jpg}
		\caption{payoff for the concrete AN offer}
	\end{figure}
	If the stock goes up $ 100 \% $ till the maturity day one would end up with $ 2645.85 \text{ USD}$ for an investment of  $ N = 1707 \text{ USD} $ . That is a gain of $ 55 \% $. If the stock price ends up to be somewhere between $ S_0 $ and the airbag level (in our case $ 0.8 * S_0 $) then a possible investor just obtains his investment. Note, that below that airbag level no complete downside protection is guaranteed, but a possible investor would nevertheless end up higher than in the case of an direct investment in the underlying stock.
	\subsection{Technical Part: How does a AN work?}
	An Airbag Note is again the combination of the three assets, we already saw before:
	\begin{itemize}
		\item A risk free asset with interest rate $ r $.
		\item Call options on the underlying stock.
		\item Put options on the underlying stock.
	\end{itemize}
	A big difference in contrast to the PPPN is that the Put options will not get bought by the bank. Quite the reverse! To design the AN the Put options have to be sold by the bank, in other words: the bank goes short on the put options. So how does it work?
	
	Again first it has to be determined the part of the investment which has to be invested in the risk free asset. Therefore we are now discounting the full initial investment by the interest rate $ r $ over the time period $ T $.  In our case consider again an investment in the US Treasure Bond of 
	\[
		N_\text{fixed} = N \cdot \exp(-rT) = \exp(-0.0011\frac{14}{12})N 
	\]
	In the case of $ N = 1707 \text{ USD} $ this is corresponding to $ 1704.81 \text{ USD} $.
	Now the bank goes short $ \frac{N}{A} $ put options with strike $ A $. 
	Considering the chosen participation rate that is just enough to buy (together with the remaining few dollars which are not invested in the risk free asset) exactly $ \frac{pN}{S_0} $ Call options with strike $ S_0 $. $ N $ is chosen in way to obtain natural numbers for $ \frac{N}{A} $ and $\frac{pN}{S_0}$. That is also the reason why the concrete number might look artificial. To look up the prices of these options we have to look up again the tables of relevant market data. 
	
	We can find for the put option \\
	\begin{tabular}{|c|c|c|c|c|c|c|}
		\hline\textbf{{LastTradeDate}} & \textbf{Strike} & \textbf{LastPrice} & \textbf{Bid} & \textbf{Ask} & \textbf{Volume} & \textbf{ImpliedVolatility}\\\hline
		2020-11-19 2:43PM EST & 67.5 & 8.6 & 8.3 & 8.5 & 12 & 48.82\%
		\\\hline
	\end{tabular}\\
	$\newline$
	and for the Call option
	
	\begin{tabular}{|c|c|c|c|c|c|c|}
		\hline\textbf{{LastTradeDate}} & \textbf{Strike} & \textbf{LastPrice} & \textbf{Bid} & \textbf{Ask} & \textbf{Volume} & \textbf{ImpliedVolatility}\\\hline
		2020-11-20 10:05AM EST & 85.0 & 18.45 & 18.15 & 18.5 & 7 & 50.39\%
		\\\hline
	\end{tabular}
	
	Note that we choose the strike of the put a little bit lower than $ S_0*0.8 = 68.28 \text{ USD} $. This will make sure that the always end up a tiny bit higher then the actual agreed  
	payoff. In the same reasoning the strike of the Call is chosen just a little bit below $ S_0 $. The reason why we have to consider those is that the strikes are not offered in a continuous way at the market but in discrete steps. This is discussed more precisely in the discussion about the profit of the bank at the end of this subsection.
	
	First check the bank is capable of buying and selling the options without making loss:
	Since the bank goes short on put options and buys call options, the bid price of the put option and the ask price of the call option has to be considered. Thus we get:
	\begin{align*}
	    c_{\text{put,bid}} = 8.3 \text{ USD}  \\
		c_{\text{call,ask}} = 18.5 \text{ USD}  
	\end{align*}
	Thus it remains after selling the put options, getting the few remaining dollars which are not invested in the risk free asset and buying:
	\begin{align*}
		N_{\text{remaining}}  &= N - N_{\text{fixed}} + \frac{N}{A}c_{\text{put,bid}} - \frac{pN}{S_0} c_{\text{call,ask}} \\
		&= N - N_{\text{fixed}} + \frac{N}{lS_0}c_{\text{put,bid}} - \frac{pN}{S_0} c_{\text{call,ask}} \\
		&= 0.0036N \\ 
	\end{align*}
	The calculation is done with Matlab to reduce rounding errors and can be found in the attached scripts (AN.m). This remaining money is one part of the margin.
	After setting up the portfolio the bank is capable of delivering the agreed payoff at the maturity date in every scenario:
	\begin{enumerate}
		\item In the case the stock ends up between $ A $ and $ S_0 $ both the bought call options and the sold put options expire without any profit or loss. Then the bank just takes $ N $ from the risk free asset, which is now there since $ \exp(-rT)N $ got invested there for the time period $ T $.
		\item If the stock ends up higher the bank has still $ N $ from the risk free asset and gets additionally $ \frac{pN}{S_0} (S_T- K_{call})$ from the call options payoff.
		\item If the stock will fall below the airbag level $ A $, the bank will use a part of the money from the risk free asset to payout the sold put options. The payoff for the customer will then be the remaining money which is $ N - \frac{N}{A}(K_{\text{put}}-S_T) $.
	\end{enumerate}
	We can write the total payoff in the following way, using that $ K_{\text{put}} > A $ and $ K_{\text{call}} < S_0 $:
	\begin{align*}
		\text{payoff}_{\text{total}} &= N - \frac{N}{A}(K_{\text{put}}-S_T)^+ + \frac{pN}{S_0}(S_T-K_{\text{call}})^+ \\
		&\geq N - \frac{N}{A}(A-S_T)^+ + \frac{pN}{S_0}(S_T-S_0)^+ \\
		&= \begin{cases}
		N - \frac{N}{A}(A - S_T) \; &\text{if } S_T < A \\
		N  \; &\text{if } A \leq S_T \leq S_0 \\
		N + \frac{pN}{S_0}(S_T-S_0) \; &\text{if }S_T > S_0 \\
		\end{cases} \\
		&= \begin{cases}
		\frac{N}{A} S_T \; &\text{if } S_T < A \\
		N  \; &\text{if } A \leq S_T \leq S_0 \\
		N + \frac{pN}{S_0}(S_T-S_0) \; &\text{if }S_T > S_0 \\
		\end{cases} \\
	\end{align*}
	Thus the bank can deliver the agreed payoff in every scenario.
	Lastly lets have a look at the additional margin which the bank gets because of the choice of $ K_\text{put} $ and $ K_{\text{call}} $. Therefore we can look at the difference of the payoffs:
	\begin{figure}[H]
		\centering
		\includegraphics[width=0.8\linewidth]{differenceAN.jpg}
		\caption{Difference of actual payoff and agreed payoff}
	\end{figure}
	There is nearly no difference in the case of $ A \leq S_T \leq S_0  $. Only at the borders there might be some difference which is smaller then in the other two cases.
	If the stock is higher than $ S_0 $ the difference is from a certain point on equal to $ 0.0023N $. In a small border region the difference might be a little bit smaller.
	If the stock is lower then $ K_{\text{put}} $ the difference is $ 0.0114N $. 
	Together with the margin obtained at time $ t=0 $ the following margin can be derived:
	\begin{align*}
		 \begin{cases}
		 \text{margin} = 0.0036N + 0.0114N \; &\text{if } S_T \leq K_\text{put} \\
		  0.0036N \leq \text{margin} \leq 0.0036N + 0.0114N  \; &\text{if } K_\text{put }<S_T < A \\
		   \text{margin} = 0.0036N  \; &\text{if } A \leq S_T \leq K_\text{call} \\
		   0.0036N \leq \text{margin} \leq 0.0036N + 0.0023N  \; &\text{if } K_\text{call }<S_T < S_0 \\
		 \text{margin} = 0.0036N + 0.0114N \; &\text{if } S_T \geq S_0
		 \end{cases}
	\end{align*}
	In the concrete case of an investment of $ N = 1707 \text{ USD} $ this is corresponding to:
	\begin{align*}
	\begin{cases}
	\text{margin} = 25.6 \text{ USD} \; &\text{if } S_T \leq K_\text{put} \\
	6.15 \text{ USD} \leq \text{margin} \leq 25.6 \text{ USD}  \; &\text{if } K_\text{put }<S_T < A \\
	\text{margin} = 6.15 \text{ USD}  \; &\text{if } A \leq S_T \leq K_\text{call} \\
	6.15 \text{ USD} \leq \text{margin} \leq 10.07 \text{ USD}  \; &\text{if } K_\text{call }<S_T < S_0 \\
	\text{margin} =  10.07 \text{ USD} \; &\text{if } S_T \geq S_0
	\end{cases}
	\end{align*}
	Additional margin can again be achieved if one is able to invest the risk free asset with a slightly higher interest rate or if its possible to get the desired call options a little bit under the ask value or sell the put options a little bit higher the bid value.
	
	
	
	\cleardoublepage
	\section{Appendix}
	\subsection{Screenshot of the market data}
	\begin{figure}[H]
		\centering
		\includegraphics[width=\linewidth]{screenshot.png}
		\caption{\label{treasure}Small extract of the relevant market data from \cite{site_yahoofinance}} 
	\end{figure}
	\subsection{Daily Treasury Yield Curve Rates}
	\begin{figure}[H]
		\centering
		\includegraphics[width=0.8\linewidth]{treasure.png}
		\caption{\label{treasure}Daily Treasury Yield Curve Rates from \cite{site_treasure}} 
	\end{figure}
	\newpage
	\bibliographystyle{abbrv}
	\bibliography{References}
	
	 
\end{document}